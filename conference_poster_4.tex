%%%%%%%%%%%%%%%%%%%%%%%%%%%%%%%%%%%%%%%%%
% baposter Landscape Poster
% LaTeX Template
% Version 1.0 (11/06/13)
%
% baposter Class Created by:
% Brian Amberg (baposter@brian-amberg.de)
%
% This template has been downloaded from:
% http://www.LaTeXTemplates.com
%
% License:
% CC BY-NC-SA 3.0 (http://creativecommons.org/licenses/by-nc-sa/3.0/)
%
%%%%%%%%%%%%%%%%%%%%%%%%%%%%%%%%%%%%%%%%%

%----------------------------------------------------------------------------------------
%	PACKAGES AND OTHER DOCUMENT CONFIGURATIONS
%----------------------------------------------------------------------------------------

\documentclass[landscape,a0paper,fontscale=0.285]{baposter} % Adjust the font scale/size here

\usepackage{graphicx} % Required for including images
\graphicspath{{figs/}} % Directory in which figures are stored

\usepackage{amsmath} % For typesetting math
\usepackage{amssymb} % Adds new symbols to be used in math mode

\usepackage{booktabs} % Top and bottom rules for tables
\usepackage{enumitem} % Used to reduce itemize/enumerate spacing
\usepackage{palatino} % Use the Palatino font
\usepackage[font=small,labelfont=bf]{caption} % Required for specifying captions to tables and figures
\usepackage{hyperref}
\usepackage{caption}
\usepackage{multicol} % Required for multiple columns
\setlength{\columnsep}{1.5em} % Slightly increase the space between columns
\setlength{\columnseprule}{0mm} % No horizontal rule between columns
\usepackage{multirow}% http://ctan.org/pkg/multirow
\usepackage{hhline}% http://ctan.org/pkg/hhline
\usepackage{tikz} % Required for flow chart
\usepackage{amsmath}
\usepackage{graphicx}
\usepackage{bm}
\usetikzlibrary{shapes,arrows} % Tikz libraries required for the flow chart in the template

\newcommand{\compresslist}{ % Define a command to reduce spacing within itemize/enumerate environments, this is used right after \begin{itemize} or \begin{enumerate}
\setlength{\itemsep}{1pt}
\setlength{\parskip}{0pt}
\setlength{\parsep}{0pt}
}

\definecolor{lightblue}{rgb}{0.53,0.81,0.98} % Defines the color used for content box headers
\DeclareMathOperator*{\argmin}{arg\,min}

\begin{document}

\begin{poster}
{
headerborder=closed, % Adds a border around the header of content boxes
colspacing=1em, % Column spacing
bgColorOne=white, % Background color for the gradient on the left side of the poster
bgColorTwo=white, % Background color for the gradient on the right side of the poster
borderColor=lightblue, % Border color
headerColorOne=white, % Background color for the header in the content boxes (left side)
headerColorTwo=lightblue, % Background color for the header in the content boxes (right side)
headerFontColor=black, % Text color for the header text in the content boxes
boxColorOne=white, % Background color of the content boxes
textborder=roundedleft, % Format of the border around content boxes, can be: none, bars, coils, triangles, rectangle, rounded, roundedsmall, roundedright or faded
eyecatcher=true, % Set to false for ignoring the left logo in the title and move the title left
headerheight=0.1\textheight, % Height of the header
headershape=roundedright, % Specify the rounded corner in the content box headers, can be: rectangle, small-rounded, roundedright, roundedleft or rounded
headerfont=\Large\bf\textsc, % Large, bold and sans serif font in the headers of content boxes
%textfont={\setlength{\parindent}{1.5em}}, % Uncomment for paragraph indentation
linewidth=2pt % Width of the border lines around content boxes
}
%----------------------------------------------------------------------------------------
%	TITLE SECTION 
%----------------------------------------------------------------------------------------
%
{\includegraphics[height=5em]{figs/usc_logo.png}} % First university/lab logo on the left
{\bf\textsc{\LARGE Semantic boundary refinement by Joint Inference from edges and regions}\vspace{.1em}} % Poster title
{\textsc{Chao ``Harry'' Yang \hspace{12pt} \parbox{0.27\textwidth}{\small University of Southern California}}} % Author names and institution

%----------------------------------------------------------------------------------------
%	OBJECTIVES
%----------------------------------------------------------------------------------------

\headerbox{\small Detecting Boundaries for Specific Objects}{name=objectives,column=0,span=1, row=0}{
\begin{minipage}[t]{1\linewidth}
\begin{tabular}{cc|cc}
\includegraphics[width=0.2\textwidth]{figs/intro/2008_001805.jpg} & 
\includegraphics[width=0.2\textwidth]{figs/intro/2008_001805_2.png} & 
\includegraphics[width=0.2\textwidth]{figs/intro/2008_007915.jpg} & 
\includegraphics[width=0.2\textwidth]{figs/intro/2008_007915_2.png} \\
\end{tabular}
\end{minipage}
We study the problem of detecting boundaries for specific classes of objects. Our approach leverages recent advances in semantic segmentation and bottom-up boundary detection. We propose a mechanism for combining multiple sources of information: predicted segmentation masks, bottom-up contours, and a novel local class-specific boundary detector. These are jointly mapped to final category-specific boundary strength estimate by a trained classifier. In experiments on VOC2010 and Microsoft COCO data sets, our method dramatically outperforms recent prior work, for some classes doubling the accuracy of boundary prediction.
}


\headerbox{The Pipeline}{name=algorithm,column=0,span=1, 
 below=objectives}{
 \vspace{0.7cm}
 Our system of category-specific boundary detection consists of the following components: 
 \begin{itemize}
 \item Semantic segmentation (DeepLab).
 \item Contour detector (UCM).
 \item Local boundary detector (ConvNets)
 \end{itemize}
A fusion component is used at the end to combine the outputs of the three components. The pipeline is illustrated in the following graph:
\begin{minipage}[t]{1\linewidth}
\includegraphics[width=1\textwidth]{figs/pipeline.png}
\end{minipage}
Our experiments use two datasets: VOC 2012 and COCO.
\vspace{0.7cm}
}

%----------------------------------------------------------------------------------------
%	INTRODUCTION
%----------------------------------------------------------------------------------------

\headerbox{\small Segmentation and Masked Boundary Baseline}{name=content,column=1,span=1, row=0}{
Segmentation and masked boundary are the first two components of our system and are also used as the baseline.
\begin{itemize}
\item Segmentation: we use the semantic segmentation contour (DeepLab) as the boundary.
\item Masked boundary: We intersect class agnostic boundary detection (UCM) with semantic segmentation mask (original segmentation dilated by 5 pixels).
\end{itemize}
The results are shown in the following figure:
\begin{minipage}[t]{1\linewidth}
  \begin{tabular}{ccc}
    \includegraphics[width=.27\linewidth]{figs/2008_001439_img}&
    \includegraphics[width=.27\linewidth]{figs/2008_001439_seg}&
    \includegraphics[width=.27\linewidth]{figs/2008_001439_ucm2}\\
    \includegraphics[width=.27\linewidth]{figs/2008_000075_img}&
    \includegraphics[width=.27\linewidth]{figs/2008_000075_seg}&
    \includegraphics[width=.27\linewidth]{figs/2008_000075_ucm2}\\
    \includegraphics[width=.27\linewidth]{figs/2008_000359_img}&
    \includegraphics[width=.27\linewidth]{figs/2008_000359_seg}&
    \includegraphics[width=.27\linewidth]{figs/2008_000359_ucm2}\\
    \includegraphics[width=.27\linewidth]{figs/2009_004581_img}&
    \includegraphics[width=.27\linewidth]{figs/2009_004581_seg}&
    \includegraphics[width=.27\linewidth]{figs/2009_004581_ucm2}\\
    input & seg & masked bdry \\
  \end{tabular}
\end{minipage}
}


\headerbox{\small Local Class-Specific Boundary Detector}{name=loss,column=1,span=1,below=content}{
We can further improve the results by training a class-specific boundary predictor from scratch. We implement it as a multi-layer convnet, trained to classify image patches as boundary or non-boundary for the class at hand:\\
\begin{minipage}[t]{1\linewidth}
  \begin{tabular}{|c|c|c|c|c|}
\hline
    &layer type& RF size & \#units&stride\\
\hline
1 & conv & 4 & 96&1\\
2 & max-pool & 2 & &2\\
3 & conv & 3 & 256& 1\\
4 & max-pool & 2 & &2\\
5 & conv & 4 & 64 &1\\
7 & ip & & 256 & \\
8 & ip & & 256 &\\
9 & ip & & 2&\\
\hline
  \end{tabular}
\end{minipage}
}

\headerbox{Information Fusion}{name=fusion,column=2,span=1}{
The three components described above each capture a different aspect of visual information. 
\begin{itemize}
\item Segmentation: may offer poor localization of the boundaries.
\item Masked boundary: low precision.
\item local detector: based on limited view of the image due to its local nature.
\end{itemize}
We use a non-linear classifier to combine them:
\[\log \text{Pr}(m=1|\mathbf{x})\,\propto\,\mathbf{w}_3^T\sigma(\mathbf{w}_2^T
\sigma(\mathbf{w}_1^T\mathbf{x}+\mathbf{b}_1)+\mathbf{b}_2)+b_3,
\]
It can be viewed as a three-layer feedforward network with fully connected layers and sigmoid activation functions!
}

\headerbox{\small Numerical Results on VOC2012}{name=weight,column=2,span=1, below=fusion}{ % This block's bottom aligns with the bottom of the conclusion block
1. Per-class accuracy comparing with STOA:\\
\footnotesize Inverse detectors: \textit{Semantic contours from inverse detectors.} \\ 
Situational detectors: \textit{Situational object boundary detection.}\normalsize\\
\begin{minipage}[t]{1\linewidth}
\includegraphics[width=.95\linewidth]{figs/barplot-voc}
\end{minipage}
2. Mean average precision:\\
\begin{minipage}[t]{1\linewidth}
  \begin{tabular}{|l|l|}
\hline
    Method & Mean precision\\\hline
    Inverse detectors& 19.9\\
    Situational detectors & 31.6\\
    \hline
    Segmentation only $S$& 45.8\\
    Masked boundary $S*B$&46.0\\
    Local patch classifier $P$ & 26.9\\
\hline
Fusion $(D_s,S*B,P)$ & 51.8\\
\hline
  \end{tabular}
 \end{minipage}
 We can see that our simple baseline already out-performs the inverse detectors and situational detectors by a large margin.
}

\headerbox{Visual Results on VOC2012}{name=drop_content,column=3,span=1}{ % This block's bottom aligns with the bottom of the 
\begin{minipage}[t]{1\linewidth}
\begin{tabular}{cccc}
  \includegraphics[width=.20\textwidth]{figs/2008_001439_img}&
  \includegraphics[width=.20\textwidth]{figs/2008_001439_final} &
\includegraphics[width=.20\textwidth]{figs/2009_000335_img}&
  \includegraphics[width=.20\textwidth]{figs/2009_000335_final}\\
\includegraphics[width=.20\textwidth]{figs/2008_000075_img}&
  \includegraphics[width=.20\textwidth]{figs/2008_000075_final} &
\includegraphics[width=.20\textwidth]{figs/2008_000359_img}&
  \includegraphics[width=.20\textwidth]{figs/2008_000359_final}\\
\includegraphics[width=.20\textwidth]{figs/2008_000271_img}&
  \includegraphics[width=.20\textwidth]{figs/2008_000271_final}&
\includegraphics[width=.20\textwidth]{figs/2009_004581_img}&
  \includegraphics[width=.20\textwidth]{figs/2009_004581_final}\\
\end{tabular}
\end{minipage}
}

\headerbox{\small Numerical Results on Microsoft COCO}{name=drop_adversarial,column=3,span=1, below=drop_content}{ 
we used the first 5,000 images from the {\tt val} set as the test set for our experiments. 
\begin{minipage}{1\linewidth}
  \begin{tabular}{|l|l|}
\hline
    Method & Mean precision\\\hline
    Segmentation only $S$& 36.9\\
    Masked boundary $S*B$&38.5\\
    Local patch classifier $P$ & 22.4\\
\hline
Fusion $(D_s,S*B,P)$ & 45.0\\
\hline
  \end{tabular}
\end{minipage}
The relative standing reflects that in VOC results, but all the numbers are lower, reflecting the
increased difficulty of COCO compared to VOC.
}


\headerbox{\small Visual Results on Microsoft COCO}{name=imagenet,column=3,span=1, below=drop_adversarial}{ 
\begin{minipage}[t]{1\linewidth}
\begin{tabular}{cccc}
\includegraphics[width=0.20\textwidth]{pascal_example/all_jpgs/2008_002965.jpg} & 
\includegraphics[width=0.20\textwidth]{pascal_example/train/2008_002965_2.png} & 
\includegraphics[width=0.20\textwidth]{pascal_example/all_jpgs/2008_003263.jpg} & 
\includegraphics[width=0.20\textwidth]{pascal_example/train/2008_003263_2.png} \\ 
\includegraphics[width=0.20\textwidth]{pascal_example/all_jpgs/2008_000835.jpg} & 
\includegraphics[width=0.20\textwidth]{pascal_example/diningtable/2008_000835_2.png} & 
\includegraphics[width=0.20\textwidth]{pascal_example/all_jpgs/2008_001077.jpg} & 
\includegraphics[width=0.20\textwidth]{pascal_example/diningtable/2008_001077_2.png} \\ 
\includegraphics[width=0.20\textwidth]{pascal_example/all_jpgs/2008_000541.jpg} & 
\includegraphics[width=0.20\textwidth]{pascal_example/sofa/2008_000541_2.png} & 
\includegraphics[width=0.20\textwidth]{pascal_example/all_jpgs/2008_002509.jpg} & 
\includegraphics[width=0.20\textwidth]{pascal_example/horse/2008_002509_2.png} \\ 
\end{tabular}
\end{minipage}

}

\headerbox{\small Class Agnostic Results}{name=paris,column=3,span=1, below=imagenet}{ 
We can also max-pooling the predictions of our category-specific detectors for each pixel, forming a class-agnostic boundary detection map:\\
\begin{minipage}{1\linewidth}
\begin{tabular}{cccc}
  \includegraphics[width=.20\textwidth]{agnostic/2010_000697.jpg}&
  \includegraphics[width=.20\textwidth]{agnostic/2010_000697.png}&
  \includegraphics[width=.20\textwidth]{agnostic/2010_003695.jpg}& 
    \includegraphics[width=.20\textwidth]{agnostic/2010_003695.png}\\
  \includegraphics[width=.20\textwidth]{agnostic/2010_004567.jpg}&
  \includegraphics[width=.20\textwidth]{agnostic/2010_004567.png}&
  \includegraphics[width=.20\textwidth]{agnostic/2010_004597.jpg}& 
    \includegraphics[width=.20\textwidth]{agnostic/2010_004597.png}\\
 \end{tabular}

\end{minipage}
}

%----------------------------------------------------------------------------------------


\end{poster}

\end{document}